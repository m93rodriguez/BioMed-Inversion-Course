\documentclass[a4paper]{article} 

\usepackage[utf8]{inputenc}
\usepackage{amsmath,amsthm}
\usepackage{bbm}
\usepackage{amsfonts}
\usepackage{amssymb}
\usepackage{graphicx}
\usepackage{color}
\usepackage{pgffor}
\usepackage{float}
\usepackage{fullpage}
\usepackage{array}
\usepackage{subcaption}
\usepackage[mathscr]{euscript}
\usepackage{bm}
\usepackage{tikz}

\usepackage{xcolor}


% Symbols - Abbreabiation of vector spaces and related quantities
\def \R {\mathbb R}

% Symbols - Vector Variables
\def \n {\mathbf n} 		% Normal vector to surface
\def \x {\mathbf x} 		% General vector
\def \r {\mathbf{r}} 		% R^3 position

\def \sori {u}				% Scalar orientation vector
\def \ori {\mathbf{\sori}} 		% Orientation vector

\def \z {\mathbf 0} 		% Vector of zeros
\def \ones {\mathbf 1} 		% Vector of ones

% Symbols - Important Linear Algebra
\def \rank {\text{rank}} 	% Vector space rank
\def \Id {\mathbb I} 	%	 Identity matrix
\def \diag {\text{diag}} 	% Main diagonal of matrix

% Symbols - shorthand names
\def \disp {\displaystyle}

% Symbols - operations
\def \dif {\mathrm{d}} % Differential
\newcommand{\difx}[2]{\frac{\dif}{\dif #1}}
\newcommand{\pdifx}[2][x]{\frac{\partial #2}{\partial #1}}
\newcommand{\inv}[1]{#1^{-1}}

\title{\LARGE \bf
	Radon Transform
}
\author{Martin Rodriguez-Vega}

\begin{document}
	
\maketitle
\thispagestyle{empty}
\pagestyle{empty}

\section{Light Propagation model}

The Radon Transform lies at the heart of X ray tomography. Consider the propagation of light in tissue where scattering is negligible
\begin{equation}
	\frac{1}{c} \pdifx[t]{I(\r,\ori)} + \ori\cdot \nabla I(\r,\ori) = -\mu_a(\r) I(\r,\ori) + s(\r,\ori)
\end{equation}	
Suppose that there are no internal sources, $s(\r,\ori) =0$, and that we consider the steady-state equilibrium, $\partial I / \partial t = 0$. Note that each for each direction $\ori$, the intensity is independent of other directions. We can find a closed form solution
\begin{equation}
	I(\r_0 + \ell \ori, \ori) = I(\r_0, \ori) \exp \left(-\int_0^\ell \mu_a(\r_0 + s\ori) \dif s \right)
\end{equation}
where $s$ [m] is a distance away from a boundary point $\r_0$ along direction $\ori$.

\section{Forward problem}

Suppose that we have detectors around the surface of the domain of interest, that the input intensity is known. We can measure then the ratio between the output/input intensities and take the log, such that
\begin{equation}
	\log\left(\frac{I(\r,\ori)}{I(\r_0, \ori)}\right) = -\int_{-\infty}^\infty \mu_a(\r_0 + s\ori) \dif  s
\end{equation}
Here, the limits are taken for all space assuming that outside the domain of interest there is no absorption. 

To get all the information along direction $\ori$, we consider all lines that do not intersect. We can do this by choosing a normal vector perpendicular to $\ori$, such that $\r_0 = t\ori^\perp$ letting $t\in(-\infty, \infty)$.

\begin{equation}
	\log\left(\frac{I(\r_0,\ori)}{I(\r, \ori)}\right) = \int_{-\infty}^{\infty} \mu_a(t\ori^\perp + s\ori) \dif s
\end{equation}
We can define the Radon transform as
\begin{equation}
	\mathcal R(\ori, t) \triangleq \int_{-\infty}^{\infty} \mu_a(t\ori^\perp + s\ori) \dif s
\end{equation}


If we choose a coordinate system $\r = (x, y)$, then we can parametrize $\ori$ using some angle $\theta$ with respect to $x$:
\begin{equation}
	\ori = \left[\begin{array}{c} \cos\theta \\ \sin\theta \end{array}\right] \quad , \quad \ori^\perp = \left[\begin{array}{c} -\sin\theta \\ \cos\theta \end{array}\right]
\end{equation}
Note that the choice of $\ori^\perp$ is not unique (can be mirrored). The Radon transform is then
\begin{equation}
	\log\left(\frac{I(\r_0,\ori)}{I(\r, \ori)}\right) = \mathcal R(\theta, t) 	\triangleq   \int_{-\infty}^\infty \mu_a(s\cos\theta - t\sin\theta, s\sin\theta + t\cos\theta) \dif s 
\end{equation}

\section{Inverse problem}

Suppose a sinogram $\mathcal (\theta, t)$ is given. We wish to recover $\mu_a(\r)$ that produced the data. Let the 2D Fourier transform of the absorption coefficient be
\begin{equation}
	M(\bm \omega) = \int_\mathbb {R}\int_\mathbb{R} \mu_a(\r) e^{-2\pi i \r \cdot \bm \omega} \dif^2 \r
\end{equation}
Similarly, the 1D Fourier transform of the sinogram for a given $\theta$ is
\begin{equation}
	\Pi(\theta, \omega) = \mathcal F\left\{ \mathcal R(\theta,t)\right\} = \int_{-\infty}^\infty \mathcal R(\theta, t) e^{-2\pi i \omega t}  \dif t = M(\omega\ori^\perp)
\end{equation}
Where the last equality comes from the Fourier slice theorem. Indeed, let $\r = t\ori^\perp + s\ori$
\begin{align}\begin{split}
	M(\omega\ori^\perp) =&  \int_\mathbb {R}\int_\mathbb{R} \mu_a(\r) e^{-2\pi i (t\ori^\perp + s\ori) \cdot (\omega \ori^\perp)} \dif t \dif s \\
	=& \int_\mathbb {R}\int_\mathbb{R} \mu_a(\r) e^{-2\pi i t\omega} \dif t \dif s \\
	=& \int_\mathbb {R} \left ( \int_\mathbb{R} \mu_a(t\ori^\perp + s\ori)  \dif s \right ) e^{-2\pi i t\omega}\dif t \\
	=& \int_\mathbb {R} \mathcal R(\ori, t) e^{-2\pi i t\omega}\dif t \\
	=& \Pi(\theta, \omega)	
\end{split}\end{align}

The most common inversion approach is the Back projection formula, which can be derived starting from the inverse Fourier transform of $M$:
\begin{equation}
	\mu_a(\r) = \int_\mathbb R\int_\mathbb R M(\bm \omega) e^{2\pi i \bm \omega \cdot \r} \dif^2 \bm\omega
\end{equation}
consider a change of variables from cartesian to polar coordinates in the frequency plane, using the angle $\theta$ and the `radius' $\omega$, such that $\bm \omega = (-\omega\sin\theta, \omega\cos\theta) = \omega\ori^\perp$:
\begin{align}
	\mu_a(\r) =& \int_0^\pi \int_\mathbb R M(\omega\ori^\perp) e^{2\pi i \omega \ori^\perp \cdot \r} |\omega |\dif \omega \dif \theta \\
	=& \int_0^\pi \int_\mathbb R \Pi(\theta, \omega) e^{2\pi i \omega t} |\omega |\dif \omega \dif \theta \\
	=& \int_0^\pi \mathcal F^{-1}\left\{\Pi(\theta, \omega) |\omega | \right\}(t) \dif \theta
\end{align}

Note that interior integral is the inverse Fourier transform of a high-pass filtered version of the spectrum of the sinogram. 

\end{document}